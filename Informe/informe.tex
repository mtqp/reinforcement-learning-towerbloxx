\documentclass[11pt, a4paper]{article}

% Configuración de márgenes de las páginas
	\usepackage{a4wide}

% Paquete de acentos para Linux
	\usepackage[utf8]{inputenc}

% Paquete para reconocer la separación en sílabas en español
	\usepackage[spanish]{babel}

% Paquetes especiales para el TP
	\usepackage{./otros/caratula}
	\usepackage{pdfpages}

% Paquete para incluir hypervinculos
	\usepackage{color}
	\usepackage{url}
	\definecolor{lnk}{rgb}{0,0,0.4}
	\usepackage[colorlinks=true,linkcolor=lnk,citecolor=blue,urlcolor=blue]{hyperref}

% Paquete para armar índices
	\usepackage{makeidx}
	\makeindex

% Más espacio entre líneas
	\parskip=1.5pt

% Opciones de enumerates
	\usepackage{enumerate}

\begin{document}

% Carátula
	\titulo{Aprendizaje por Refuerzos}
	\fecha{2012}
	\materia{Aprendizaje por Refuerzos}
	\integrante{Mariano De Sousa}{??}{marian_sabianaa@hotmail.com}
	\integrante{Mariano Bianchi}{92/08}{marianobianchi08@gmail.com}
	\integrante{Pablo Brusco}{527/08}{pablo.brusco@gmail.com}
	\maketitle

\section{Ambiente}
Detalle sobre ambiente utilizado:

La grúa es un objeto que tiene un movimiento lineal entre dos posiciones (en este caso -49 y +49), a velocidad constante (1) se mueve en una u otra dirección, de donde se tirará el nuevo piso a agregar. 

Por otro lado, el edificio, posee un movimiento pendular que va variando en velocidad dependiendo de un factor de desbalanceado del mismo. La velocidad varía en un rango discreto de -5 a 5 inclusive el cual determina su posición (entre -49 y 49 inclusive).

El factor de desbalanceado es un número entre 0 y 1, que al sobrepasar este umbral conciderámos que el edificio se cae. 






\end{document}